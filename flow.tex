%%%%%%%%%%%%%%%%%%%%%%%%%%%%%%%%%%%%%%%%%%%%%%%%%%%%%%%%%%%%
%Capítulo: Controle de fluxo
%%%%%%%%%%%%%%%%%%%%%%%%%%%%%%%%%%%%%%%%%%%%%%%%%%%%%%%%%%%%

\chapter{Controle de fluxo}

Um programa de computador é uma sequência de instruções. Normalmente as instruções são executadas na ordem em que aparecem no programa. Porém, isso permite apenas a criação de programas muito simples. 

Imagine que tenhamos por exemplo uma matriz com centenas de valores. Para processar cada item dessa matriz, precisamos executar centenas de instruções. Se a instrução for a mesma para todos os itens da matriz, podemos facilitar nossa vida criando um programa bem mais simples se colocarmos essa instrução dentro de um {\it laço}, ao invés de ter uma cópia desta mesma instrução para cada item da matriz.

Outro caso em que o controle de fluxo é útil é quando uma instrução é executada em apenas alguns casos. Imagine que tempos um banco de dados com milhões de valores de CPF e queremos imprimir apenas o nome do portador de um certo CPF. Podemos usar uma instrução do tipo {\tt if} que, para cada CPF lido, testa se é igual ao CPF desejado e imprime o nome do portador somente em caso afirmativo.

%%%%%%%%%%%%%%%%%%%%%%%%%%%%%%%%%%%%%%%%%%%%%%%%%%%%%%%%%%%%
%Seção: if / else

\section{if / else}

Uma instrução do tipo \IF\ executa um bloco de instruções somente se uma dada condição é verdadeira. Opcionalmente pode ser complementada com uma instrução \ELSE, que executa outro bloco caso a dada condição seja falsa.

\begin{itemize}

\item A forma geral de uma instrução \IF\ é a seguinte:

{\tt 
if (<expressão condicional>)                     \\
\{                                               \\
\TAB <bloco executado se a condição for verdadeira>  \\
\}                                               \\
}

\item A forma geral de uma instrução \IF/\ELSE\ é a seguinte:

{\tt 
if (<expressão condicional>)                     \\
\{                                               \\
\TAB <bloco executado se a condição for verdadeira>  \\
\}                                               \\
else                                             \\
\{                                               \\
\TAB <bloco executado se a condição for falsa>   \\
\}                                               \\
}

\item Observe que a expressão condicional só aparece uma vez, junto ao \IF, jamais junto ao \ELSE.

\item A expressão condicional pode ser uma expressão lógica, de comparação, ou uma operação aritmética com inteiros. Nao usar expressões aritméticas com \FLOAT\ ou \DOUBLE.
\item O bloco do \IF\ é executado se a expressão condicional der como resultado \TRUE\ ou diferente de zero.
\item O bloco do \ELSE\ é executado se a expressão condicional der como resultado \FALSE\ ou igual a zero.

\end{itemize}

%%%%%%%%%%%%%%%%%%%%%%%%%%%%%%%%%%%%%%%%%%%%%%%%%%%%%%%%%%%%
%Seção: for

\section{for}
\label{sec:for}

Palavra reservada para criar instruções de iteração, ou seja, laços. Um laço é um bloco que é executado várias vezes.

\begin{itemize}

\item A forma geral de uma instrução \FOR\ é a seguinte:

{\tt 
for (<inicialização>; <condição de continuidade>; <incremento>) \\
\{                                                            \\
\TAB <bloco a ser executado várias vezes>                     \\
\}                                                            \\
}

\item Na inicialização, geralmente uma variável de controle é declarada e inicializada.
\item Variável de controle é a que indica quando o laço deve parar.
\item Na condição de continuidade, se testa se o laço deve continuar. Geralmente isso é feito comparando a variável de controle com algum valor.
\item No incremento, a variável de controle é incrementada. Pode também ser decrementada se necessário.

\end{itemize}

%%%%%%%%%%%%%%%%%%%%%%%%%%%%%%%%%%%%%%%%%%%%%%%%%%%%%%%%%%%%
%Seção: while

\section{while}
\label{sec:while}

Faz exatamente o mesmo que o \FOR, ou seja, serve para criar laços. A diferença está na sintaxe.

\begin{itemize}

\item A forma geral de uma instrução \WHILE\ é a seguinte:

{\tt 
<inicialização>;                                              \\ 
while (<condição de continuidade>)                            \\
\{                                                            \\
\TAB <bloco a ser executado várias vezes>                     \\
\TAB <incremento>;                                            \\
\}                                                            \\
}

\item A inicialização, condição de continuidade e incremento funcionam exatamente da mesma maneira que no \FOR, conforme explicado na Seção~\ref{sec:for}

\end{itemize}


