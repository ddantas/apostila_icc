%%%%%%%%%%%%%%%%%%%%%%%%%%%%%%%%%%%%%%%%%%%%%%%%%%%%%%%%%%%%
%Capítulo: Funções
%%%%%%%%%%%%%%%%%%%%%%%%%%%%%%%%%%%%%%%%%%%%%%%%%%%%%%%%%%%%

\chapter{Funções}

Funções são blocos de instruções que executam uma certa tarefa. 
Podem receber dados de entrada e retornar dados de saída. 
Podemos dizer que a definição de uma função tem cinco partes: identificador ou nome, tipo de dado de retorno, lista de parâmetros, bloco de instruções e valor de retorno.

\begin{itemize}

\item Uma declaração de função tem a seguinte forma geral:

{\tt
<Tipo de dado> <Identificador> (<Lista de parâmetros>)\\
\{\\
\TAB <Bloco de código>\\
\TAB return <Valor de retorno>;\\
\}\\
}

\item Identificador é o nome através do qual chamamos a função. Alguns identificadores de funções já vistas são por exemplo \PRINTF, \SCANF, \FGETS, \SIN\ e \COS.

\item Tipo de dado de retorno é o tipo de dado que a função retorna. Como já foi visto na Seção~\ref{sec:expr}, uma chamada a uma função é uma expressão elementar, e pode ser usada da mesma maneira que uma variável ou valor de mesmo tipo.

\begin{lstlisting}
  double x = sin(a);
  printf("%f\n", x);
\end{lstlisting}

  Observe que \PRINTF\ imprimirá o valor de tipo \DOUBLE\ retornado pela função \SIN. De maneira alternativa, poderíamos colocar a chamada \SIN\ diretamente como parãmetro de \PRINTF.

\begin{lstlisting}
  printf("%f\n", sin(a));
\end{lstlisting}

\item Lista de parâmetros são os dados de entrada da função. Cada parâmetro é uma variável cujo valor é definido na chamada da função. Cada item da lista tem o mesmo formato de uma declaração de variável, ou seja, é um tipo de dado seguido de um identificador. Se uma função recebe mais de um parâmetro, eles devem ser separados por vírgula.

\item Bloco de código é o código executado quando a função é chamada.

\item Valor de retorno é o valor retornado para a função que fez a chamada. Uma chamada a função pode fazer parte de uma expressão e tem o mesmo efeito que seu valor de retorno.

\end{itemize}

No exemplo abaixo podemos ver uma declaração bem simples de função. Essa função não recebe parâmetros e nem retorna valor. Simplesmente imprime uma mensagem e retorna. Logo após a declaração, vemos um exemplo de função \MAIN\ que a chama.

\begin{lstlisting}
#include <stdio.h>
void cumprimenta(void)
{
  printf("Bom dia");
}

int main(void)
{
  cumprimenta();
}
\end{lstlisting}

Abaixo temos outro exemplo de função que não retorna valores. Essa função recebe um número de tipo \DOUBLE\ e o imprime usando notação científica.

\begin{lstlisting}
#include <stdio.h>
void imprimed(double val)
{
  printf("%.10E", val);
}

int main(void)
{
  imprimed(299792458.0);
  printf("\n");
}
\end{lstlisting}

Vale notar que o tipo de dado de retorno da função não tem relação com os tipos dos parãmetros. Podem ser completamente diferentes como no exemplo abaixo, que recebe \FLOAT\ como entrada e retorna \INT. A expressão {\tt (int) } executa um {\it typecast}, ou seja, uma conversão de tipo. 

\begin{lstlisting}
#include <stdio.h>
int arredonda(float val)
{
  return (int) (val+0.5);
}

int main(void)
{
  printf("Pi eh aproximadamente %d\n", arredonda(3.14159265));
}
\end{lstlisting}

%%%%%%%%%%%%%%%%%%%%%%%%%%%%%%%%%%%%%%%%%%%%%%%%%%%%%%%%%%%%
%Seção: Passagem de parâmetros por valor

\section{Passagem de parâmetros por valor}

Quando chamamos uma função, os valores dos parãmetros são copiados para as variáveis definidas na lista de parâmetros. Podemos modificar esses valores no interior da função sem que isso tenha efeito fora dela.

\begin{lstlisting}
#include <stdio.h>
int dobra(int n)
{
  n = 2 * n;
  return n;
}

int main(void)
{
  int a = 5;
  int b = dobro(a);
  printf("O dobro de %d eh %d\n", a, b);
}
\end{lstlisting}

O valor de {\tt a} é copiado para o parâmetro {\tt n} da função. {\tt n} recebe 10 mas {\tt a} permanece 5.

%%%%%%%%%%%%%%%%%%%%%%%%%%%%%%%%%%%%%%%%%%%%%%%%%%%%%%%%%%%%
%Seção: Passagem de parâmetros por referência

\section{Passagem de parâmetros por referência}

Esse tipo de passagem de parâmetro é necessário quando a função deve ser capaz de alterar alguma variável fora de seu escopo. Em outras palavras, quando precisamos que a função altere alguma variável da função que a chamou. Um exemplo de passagem de parâmetro por referência é o que fazemos ao chamar \SCANF.

Nesse tipo de passagem de parâmetro, ao invés de passarmos um valor, passamos um ponteiro, ou seja, uma referência à posição de memória que gostaríamos de alterar.

No exemplo abaixo, a função {\tt troca} é implementada. Ela troca os valores de duas variáveis da função que a chamou. Isso só é possível usando passagem de parâmetro por referência.

\begin{lstlisting}
#include <stdio.h>
int troca(int *a, int *b)
{
  int aux = *a;
  *a = *b;
  *b = aux;
}

int main(void)
{
  int a = 5;
  int b = 3
  printf("Antes: x = %d, y = %d\n", x, y);
  troca(&x, &y);
  printf("Depois: x = %d, y = %d\n", x, y);
}
\end{lstlisting}


