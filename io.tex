%%%%%%%%%%%%%%%%%%%%%%%%%%%%%%%%%%%%%%%%%%%%%%%%%%%%%%%%%%%%
%Capítulo: Entrada e saída
%%%%%%%%%%%%%%%%%%%%%%%%%%%%%%%%%%%%%%%%%%%%%%%%%%%%%%%%%%%%

\chapter{Entrada e saída}

Entrada e saída se refere à comunicação do programa, tanto com o usuário quanto com arquivos em disco. 
Duas funções que podem ser usadas para fazer entrada e saída com o usuário pelo console são \PRINTF\ e \SCANF. 
Para fazer entrada e saída com arquivos, pode-se usar \FPRINTF\ e \FSCANF. 
Para usar qualquer uma dessas funções, adicionar a seguinte linha ao início do código:

\begin{lstlisting}
#include <stdio.h> 
\end{lstlisting}


%%%%%%%%%%%%%%%%%%%%%%%%%%%%%%%%%%%%%%%%%%%%%%%%%%%%%%%%%%%%
%Seção: printf

\section{printf}
\label{sec:printf}

\PRINTF\ não é uma palavra reservada como \WHILE, e sim uma função. \PRINTF\ espera como primeiro parâmetro uma string, como nos exemplos abaixo.

\begin{lstlisting}
printf("Hello world\n");
printf("bom dia");
printf("Pressione enter para continuar...\n");
\end{lstlisting}

Observe que strings ficam entre aspas duplas. Ao final de algumas das strings acima está o caractere especial '\textbackslash n'. Esse caractere representa um pulo para a linha seguinte. Sempre adicionar esse caractere ao final quando a string contiver uma mensagem que deve aparecer isolada em uma linha. 


\begin{table}
\centering
  \begin{tabular}{|l|l|l|}
    \hline
        Caractere               & Valor ASCII & Representação \\
    \hline
        Pula linha              & 10          & \verb|\n|     \\
        Tabulação               & 9           & \verb|\t|     \\
        {\it Beep}              & 7           & \verb|\a|     \\
        {\it Backslash}         & 92          & \verb|\\|     \\
        Aspas duplas            & 34          & \verb|\"|     \\
        Símbolo de percentagem  & 37          & \verb|%%|     \\
    \hline
  \end{tabular}
  \caption{Caracteres especiais do C}
  \label{tab:chars}
\end{table}


Além do pulo de linha, existem outros caracteres especiais que \PRINTF\ reconhece, como pode ser visto na Tabela~\ref{tab:chars}

\pagebreak

Para ser mais versátil, \PRINTF\ pode imprimir valores de expressões no meio da mensagem. Para isso é necessário colocar na string um especificador de formato no ponto em que o valor deve aparecer, e adicionar como parâmetro a expressão a ser impressa. Especificadores de formato são precedidos pelo símbolo \%\ e servem para definir como interpretar o valor da expressão, ou seja, se é um valor inteiro, real, caractere etc. Observe o exemplo abaixo.

\begin{lstlisting}
printf("Um inteiro ocupa %d bytes\n", sizeof(int));
printf("PI vale %f\n", 3.141592653589793238462);
\end{lstlisting}

Cada uma das strings possui um especificador de formato. A primeira possui um {\tt \%d} 
para imprimir um número inteiro e a segunda possui um {\tt \%f}
para imprimir um número real.
A saída do exemplo acima será a seguinte:\\
\\
{\tt
Um inteiro ocupa 4 bytes\\
PI vale 3.141593\\
}

Por padrão o {\tt \%f} imprime o número real com seis casas decimais, mas esse comportamento pode ser modificado colocando logo depois do símbolo {\tt \%} um ponto seguido do número de casas decimais desejado, como a seguir.

\begin{lstlisting}
printf("PI vale %.10f\n", 3.141592653589793238462);
\end{lstlisting}

cuja saída será\\
\\
{\tt
PI vale 3.1415926536\\
}

O formato de números inteiros também pode ser modificado. Colocando um número {\tt n} de dígitos entre o {\tt \%} e o {\tt d}, fará cada inteiro ocupar {\tt n} espaços. Isso é útil para se imprimir tabelas onde o dígito menos significativo deve ficar alinhado à direita. Para completar o inteiro com zeros à esquerda, colocar um zero entre o {\tt \%} e o número de dígitos completará o número impresso com zeros à esquerda até que ocupe o tamanho desejado. Observe os exemplos abaixo e compare suas saídas.

\pagebreak

\begin{lstlisting}
printf("Um inteiro ocupa %d bytes\n", sizeof(int));
printf("Um inteiro ocupa %6d bytes\n", sizeof(int));
printf("Um inteiro ocupa %06d bytes\n", sizeof(int));
\end{lstlisting}

terão como saída respectivamente

\begin{verbatim}
Um inteiro ocupa 4 bytes
Um inteiro ocupa      4 bytes
Um inteiro ocupa 000004 bytes
\end{verbatim}


\begin{table}
\centering
  \begin{tabular}{|l|l|l|}
    \hline
        Saída                      & Especificador & Exemplo            \\
    \hline
        Inteiro decimal                      & \%d           & 45                 \\
        Inteiro decimal \LONGINT             & \%ld          & 12345678901234     \\
        Inteiro hexadecimal                  & \%x           & f3 ou F3           \\
                                             & \%X           & f3 ou F3           \\
        Número real \FLOAT                   & \%f           & 3.141593           \\
        Número real \DOUBLE                  & \%lf          & 3.141593           \\
        Número real em notação científica    & \%e           & 1.000000e+09       \\
                                             & \%E           & 1.000000E+09       \\
        Caractere                            & \%c           & a                  \\
        String                               & \%s           & Hello              \\
        Endereço de ponteiro                 & \%p           & 0x7fffec2e33a8     \\
    \hline
  \end{tabular}
  \caption{Especificadores de formato do C}
  \label{tab:format}
\end{table}

Em uma mesma string pode haver vários especificadores de formato, bastando que, para cada um dos especificadores, deve haver um valor correspondente como parâmetro. Observe o exemplo abaixo. Existem dois especificadores de formato. O primeiro será substituído pelo valor de {\tt val1} e o segundo, pelo valor de {\tt val2}.

\begin{lstlisting}
int val1 = 33;
float val2 = 1.23;
printf("A variavel inteira vale %d e a variavel real vale %f\n", val1, val2);
\end{lstlisting}





%%%%%%%%%%%%%%%%%%%%%%%%%%%%%%%%%%%%%%%%%%%%%%%%%%%%%%%%%%%%
%Seção: scanf

\section{scanf}
\label{sec:scanf}

A chamada a \SCANF\ é semelhante à chamada ao \PRINTF. O primeiro parâmetro também é uma string, mas deve possuir um especificador de formato. Os demais parãmetros são os endereços das posições de memória onde os dados serão gravados. É recomendável porém usar uma chamada \SCANF\ para cada valor a ser lido. Para passar o endereço de memória, é necessário preceder a variável pelo símbolo {\tt \%}.

\begin{lstlisting}
int i;
scanf("%d", &i);
float x;
scanf("%f", &x);
char letra;
scanf("%c", &letra);
\end{lstlisting}

\begin{itemize}

\item \SCANF\ retorna o número de valores lidos com sucesso. Idealmente deve-se usar esse valor para testar se a leitura foi bem sucedida.

\item Segue um exemplo de código com \SCANF.

\begin{lstlisting}
#include <stdio.h> 
#include <cstdlib> 
int main (void)
{
  int result;
  float userfloat;
  printf("Digite um numero real:\n");
  result = scanf("%f", &userfloat);
  printf("result = %d, real digitado = %f\n", result, userfloat);
  system("pause");
}
\end{lstlisting}

\item No exemplo acima, {\tt userfloat} armazena o real digitado pelo usuário. A variável {\tt result} será igual a 1 caso a leitura seja bem sucedida, e 0 caso contrário.

\end{itemize}

%%%%%%%%%%%%%%%%%%%%%%%%%%%%%%%%%%%%%%%%%%%%%%%%%%%%%%%%%%%%
%Seção: sscanf

\section{sscanf}
\label{sec:sscanf}

\SCANF\ pode não funcionar corretamente quando o usuário digita valores de tipos diferentes do esperado. Uma maneira de contornar isso é armazenando a string em uma variável intermediária antes de fazer a leitura. Nesse caso, a leitura é feita com \SSCANF, uma versão alternativa de \SCANF\ que lê dados de strings.

\begin{lstlisting}
#include <stdio.h> 
#include <cstdlib> 
int main (void)
{
  int result;
  float userfloat;
  char line[256];
  printf("Digite um numero real:\n");
  gets(line);
  result = sscanf(line, "%f", &userfloat);
  printf("result = %d, real digitado = %f\n", result, userfloat);
  system("pause");
}
\end{lstlisting}

\begin{itemize}

\item No exemplo acima, podemos observar uma nova variável chamada {\tt line}, que armazenará o texto digitado pelo usuário. \GETS\ lê o texto digitado e o armazena em {\tt line}, que é passado como parâmetro para \SSCANF. As demais partes do código permanecem como no exemplo da Seção~\ref{sec:scanf}.

\end{itemize}


%%%%%%%%%%%%%%%%%%%%%%%%%%%%%%%%%%%%%%%%%%%%%%%%%%%%%%%%%%%%
%Seção: fscanf

\section{fscanf}
\label{sec:fscanf}

\FSCANF é uma função que pode ser usada para leitura de dados de arquivos. Recebe um parâmetro a mais que \SCANF: o ponteiro para o arqivo a ser lido.

\begin{itemize}

\item Para obter o ponteiro para o arquivo, usar a função \FOPEN. No exemplo abaixo, {\tt fp} é a variável que armazenará o ponteiro para o arquivo. O primeiro parâmetro é uma string que contém o nome do arquivo a ser aberto. Neste exemplo é {\tt arquivo.txt} mas poderia ser qualquer outro. O segundo parâmetro indica o tipo de acesso desejado, ou seja, leitura ou escrita. Nesse caso contém uma letra {\tt r} para indicar que o arqivo será aberto para leitura. Usar a letra {\tt w} para escrita. \FOPEN\ retorna zero em caso de erro, ou seja, se o arquivo não existir ou se não houver permissão para o acesso desejado.

\begin{lstlisting}
FILE* fp = fopen("arquivo.txt", "r");
\end{lstlisting}

\item Uma maneira de testar se o arquivo foi aberto corretamente é mostrada a sqguir. Uma mensagem mais esclarecedora para o usuário, e portanto mais útil, pode incluir o nome do arquivo.

\begin{lstlisting}
  if (fp == 0)
  {
    printf("Erro ao abrir arquivo\n");
    return 1;
  }
\end{lstlisting}


\item Uma vez que o arquivo esteja aberto, chamar \FSCANF\ com o ponteiro retornado por \FOPEN\ como primeiro parâmetro. Como \SCANF, recebe também uma string com o especificador de formato, e a variável onde o valor lido deve ser armazenado. Essa variável deve ser passada por referência, ou seja, é necessário usar o símbolo {\tt \&} antes do identificador da variável para indicar que é passado seu endereço como parâmetro.

\begin{lstlisting}
int i;
fscanf(fp, "%d", &i);
float x;
fscanf(fp, "%f", &x);
char letra;
fscanf(fp, "%c", &letra);
\end{lstlisting}

\item Segue um programa exemplo que lê dois números inteiros de um arquivo. Cada número deve estar em uma linha diferente no arquivo.

\begin{lstlisting}
#include <stdio.h> 
int main (void)
{
  int a, b;
  FILE* fp = fopen("arquivo.txt", "r");
  if (fp == 0)
  {
    printf("Erro ao abrir arquivo\n");
    return 1;
  }
  fscanf(fp, "%d", &a);
  printf("Valor de a = %d\n", a);
  fscanf(fp, "%d", &b);
  printf("Valor de b = %d\n", b);
  fclose(fp);
}
\end{lstlisting}

\item Abaixo está uma versão alternativa que usa \FGETS\ para ler cada linha do arquivo por completo antes de tentar extrair dela o número. Observe que a função \FGETS\ é usada no lugar da \GETS, da Seção~\ref{sec:sscanf}. \

\begin{lstlisting}
#include <stdio.h> 
int main (void)
{
  int MAX_LEN = 256
  int a, b;
  FILE* fp = fopen("arquivo.txt", "r");
  if (fp == 0)
  {
    printf("Erro ao abrir arquivo\n");
    return 1;
  }
  char line[MAX_LEN];
  fgets(line, MAX_LEN, fp);
  sscanf(line, "%d", &a);
  printf("Valor de a = %d\n", a);
  fgets(line, MAX_LEN, fp);
  sscanf(line, "%d", &b);
  printf("Valor de b = %d\n", b);
  fclose(fp);
}
\end{lstlisting}


\item Idealmente, deve-se testar se cada \FSCANF\ e \FGETS\ fez uma leitura bem sucedida, como no trecho de código abaixo. \FGETS\ recebe três parâmetros: um vetor de caracteres ou string, o tamanho máximo da string e o ponteiro para o arquivo. Retorna zero em caso de erro.

\begin{lstlisting}
  char* result1 = fgets(line, MAX_LEN, fp);
  int result2 = sscanf(line, "%d", &n);
  if (result1 != 0 && result2 == 1)
  {
    printf("Valor lido = %d\n", result, n);
  }
  else
  {
    printf("Valor invalido\n");
  }
\end{lstlisting}

\item Ao final do programa, após fazer todas as leituras necessárias, fechar o arquivo com \FCLOSE, como na linha de código a seguir. Isso salva as escritas no arquivo e o libera para que outros programas possam acessá-lo.

\begin{lstlisting}
  fclose(fp);
\end{lstlisting}

\end{itemize}


