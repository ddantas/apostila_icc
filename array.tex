%%%%%%%%%%%%%%%%%%%%%%%%%%%%%%%%%%%%%%%%%%%%%%%%%%%%%%%%%%%%
%Capítulo: Vetores
%%%%%%%%%%%%%%%%%%%%%%%%%%%%%%%%%%%%%%%%%%%%%%%%%%%%%%%%%%%%

\chapter{Vetores}

Vetores são sequências de valores do mesmo tipo referenciados por um mesmo nome. Um elemento específico da sequência é acessado por meio de um índice inteiro não negativo.

Na linguagem C, os vetores começam com o índice zero. Um vetor com {\tt n} posições terá como índices válidos os inteiros de zero a {\tt n-1}.

%%%%%%%%%%%%%%%%%%%%%%%%%%%%%%%%%%%%%%%%%%%%%%%%%%%%%%%%%%%%
%Seção: Vetores unidimensionais

\section{Vetores unidimensionais}

Uma declaração de vetor tem o seguinte formato:

\NEWLINE
{\tt
<Tipo de dado> <Identificador> [ <Tamanho> ];
}
\NEWLINE

A expressão elementar abaixo funciona como uma variável comum, podendo ser usada em expressões aritméticas, de comparação, atribuição, ou como parâmetro em chamadas de funções.

\NEWLINE
{\tt
<Identificador> [ <Posição> ]
}
\NEWLINE

Uma atribuição de valor a uma certa posição do vetor, por exemplo, tem o seguinte formato:

\NEWLINE
{\tt
<Identificador> [ <Posição> ] = <Valor>;
}
\NEWLINE

%%%%%%%%%%%%%%%%%%%%%%%%%%%%%%%%%%%%%%%%%%%%%%%%%%%%%%%%%%%%
%Seção: Matrizes

\section{Matrizes}

Uma matriz é um vetor bidimensional. Sua declaração é semelhante à de um vetor unidimensional.

\NEWLINE
{\tt
<Tipo de dado> <Identificador> [ <Linhas> ][ <Colunas> ];
}
\NEWLINE


Para acessar um item individual de uma matriz, usar o formato abaixo:

\NEWLINE
{\tt
<Identificador> [ <Linha> ][ <Coluna> ]
}
\NEWLINE


